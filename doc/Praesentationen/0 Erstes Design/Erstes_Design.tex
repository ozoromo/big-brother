\documentclass[xcolor=x11names,table]{beamer}

\usepackage{myDefaultPackageSetup/Beamer/myDefaultsBeamer}
\ProvidesPackage{myTheme1}

\usetheme{Madrid}
\setbeamercolor{title}{fg=white}
\setbeamercolor*{subtitle}{fg=white}

\setbeamercolor{section title}{fg=white,bg=blue!30}
\setbeamercolor{subsection title}{fg=white,bg=blue!30}
\setbeamercolor{subsubsection title}{fg=white,bg=blue!30}

\beamertemplatenavigationsymbolsempty
\setbeamertemplate{frame numbering}[fraction]
\AtBeginSection[]{
  \begin{frame}
  \vfill
  \centering
  \begin{beamercolorbox}[sep=8pt,center,shadow=true,rounded=true]{section title}
    \usebeamerfont{section title}\insertsectionhead\par%
  \end{beamercolorbox}
  \vfill
  \end{frame}
}
\AtBeginSubsection{
    \begin{frame}
    \centering
    {\usebeamerfont{subsection name}\usebeamercolor[fg]{subsection %
    name}\subsectionname~\insertsubsectionnumber}%
    \vskip1em\par
    \begin{beamercolorbox}[sep=4pt,center,shadow=true,rounded=true]{subsection title}
      \usebeamerfont{subsection title}\insertsubsection\par
    \end{beamercolorbox}
    \end{frame}
}
\AtBeginSubsubsection{
    \begin{frame}
    \centering
    {\usebeamerfont{subsubsection name}\usebeamercolor[fg]{subsubsection %
    name}\subsubsectionname~\insertsubsubsectionnumber}%
    \vskip1em\par
    \begin{beamercolorbox}[sep=4pt,center,shadow=true,rounded=true]{susubsection title}
      \usebeamerfont{subsection title}\insertsubsubsection\par
    \end{beamercolorbox}
    \end{frame}
}
% Section numbering
\setbeamertemplate{section in toc}[sections numbered]
\setbeamertemplate{subsection in toc}[subsections numbered]
\makeatletter
\setbeamertemplate{section page}{
  \centering
  \begin{minipage}{22em}
    \raggedright
    \usebeamercolor[fg]{section title}
    \usebeamerfont{section title}
    \thesection. \insertsectionhead\\[-1ex]
    \usebeamertemplate*{progress bar in section page}
    \par
    \ifx\insertsubsectionhead\@empty\else%
      \usebeamercolor[fg]{subsection title}%
      \usebeamerfont{subsection title}%
      \thesection. \thesubsection.\insertsubsectionhead
    \fi
  \end{minipage}
  \par
  \vspace{\baselineskip}
}
\makeatother


%%%%%%%%%%%%%%%%%%%%%%%%%%%%%%%%%%%%%
%       Show notes on pympress      %
%%%%%%%%%%%%%%%%%%%%%%%%%%%%%%%%%%%%%
%\usepackage{pgfpages}
%\setbeamertemplate{note page}[plain]
%\setbeameroption{show notes on second screen=right}
%%\setbeameroption{show only notes}

\title[\textcolor{white}{Erstes Design}]
{Gruppe A3 - Big Brother: Erstes Design}
\author{Gruppe A3}
\institute[TU Berlin]{TU Berlin}
\date{\today}

\begin{document}
\small
\begin{frame}
\titlepage
\end{frame}

%\begin{frame}{<++>}
%\tableofcontents
%\end{frame}

\begin{frame}{Spezifikation}
\begin{itemize}
    \item Aufbauend auf SS21 und SS22
    \narrow Es wird ausgewählt, welches implementation genutzt
    wird
    \item Angestrebte Funktionalitäten:
    \begin{itemize}
        \item Gestenerkennung mit Live Video 
        \item Regulierung von Einstellungen mit Gesten
        \narrowB Lautstärke
        \item Nebenläufiger Webseitenzugriff
        \item Gesten als Passwort nutzen
    \end{itemize}
\end{itemize}
\end{frame}

\section{Organisation}%
\label{sec:Organisation}

\begin{frame}{Projektmanagement}
\begin{itemize}
    \item Gruppenaufteilung nach Rollen:
    \begin{itemize}
        \item Rollen: Frontend, Logik, Backend
        \item Gruppenverantwortlicher (eine Person):
        \begin{itemize}
            \item prüfen der Schnittstelle
            \item Absprache mit anderen Gruppen
            \item adequate Dokumentation
        \end{itemize}
        \item Kommunikation in kleineren Gruppen
    \end{itemize}
    \item Kommunikation über Discord
\end{itemize}
\end{frame}

\begin{frame}{Teamaufteilung}
\begin{itemize}
    \item Als Namen werden die Discord-Handles genutzt, wobei die
    Verantwortlichen \txtbf{fett} markiert sind
    \item Frontend: \txtbf{maltrim}, Leland, GOAT
    \item Logik: \txtbf{Nimando}, Max1.1, erik, milosz, Mati,
    adamm, minhdarine
    \item Backend: \txtbf{Mike}, Julian, ic4russ,CrazyFisch
\end{itemize}
\end{frame}

\begin{frame}{Tools}
\begin{itemize}
    \item Sprache: Python
    \item Frontend: 
    \href{https://flask.palletsprojects.com/en/2.2.x/}{Flask}, 
    HTML, CSS, JS
    \item Datenbank: \href{https://www.mongodb.com/}{MongoDB}
    \item Gesichtserkennung und Gestenerkennung: 
    \href{https://pypi.org/project/opencv-python/}{openCV},
    \href{https://pytorch.org/}{TensorFlow},
    \href{https://pytorch.org/}{PyTorch}
\end{itemize}
\end{frame}


\section{Planung - Projektablauf}
\begin{frame}{1. Milestone}
\begin{itemize}
    \item Verstehen und umsetzen des Codes letzter vergangener
    Jahre
    \item Definieren und Dokumentieren von Schnittstellen
    \begin{tab}{l|l|l}
    \txttH{Frontend} & \txttH{Logik} & \txttH{Backend} \breakT
    %
    \begin{minipage}[t]{0.3\textwidth}
    \begin{itemize}
        \item Überarbeiten der Webseite
    \end{itemize}
    \end{minipage} & 
    %
    \begin{minipage}[t]{0.3\textwidth}
    \begin{itemize}
        \item Gesichtserkennung ausprobieren
        \item Aufteilung in Subgruppen
    \end{itemize}
    \end{minipage} & 

    \begin{minipage}[t]{0.3\textwidth}
    \begin{itemize}
        \item Datenbank zum laufen bekommen
        \item Anfangen passende Trainings- und Testdaten zu
        recherchieren
    \end{itemize}
    \end{minipage}
    %
    \end{tab}
\end{itemize}
\end{frame}

\begin{frame}{2. Milestone}
\begin{tab}{l|l|l}
\txttH{Frontend} & \txttH{Logik} & \txttH{Backend} \breakT
%
\begin{minipage}[t]{0.3\textwidth}
\begin{itemize}
    \item Verbindung mit Daten im Backend
    \item Live Kamera-Aufnahmen an Logik übertragen
    \item Design verbessern
\end{itemize}
\end{minipage} & 
%
\begin{minipage}[t]{0.3\textwidth}
\begin{itemize}
    \item Zuordnen von Gesichtern zu Personen
    \item Gestenerkennung
\end{itemize}
\end{minipage} & 

\begin{minipage}[t]{0.3\textwidth}
\begin{itemize}
    \item Daten(-bank) strukturieren
    \item Erstellen wichtiger Datenbankanfragen
\end{itemize}
\end{minipage}
%
\end{tab}
\end{frame}

\begin{frame}{3. Milestone}
\begin{tab}{l|l|l}
\txttH{Frontend} & \txttH{Logik} & \txttH{Backend} \breakT
%
\begin{minipage}[t]{0.3\textwidth}
\begin{itemize}
    \item Zugriff von mehreren Nutzern auf Funktionalitäten
\end{itemize}
\end{minipage} & 
%
\begin{minipage}[t]{0.3\textwidth}
\begin{itemize}
    \item an ``Fingergesten'' Passwörter erkennen
    \item Lautstärke mit Gesten kontrollieren
\end{itemize}
\end{minipage} & 

\begin{minipage}[t]{0.3\textwidth}
\begin{itemize}
    \item Umgang mit parallelen Datenbankanfragen
    \item Andere Gruppen helfen 
\end{itemize}
\end{minipage}
%
\end{tab}
\end{frame}

\begin{frame}{Sonstiges zum Projektablauf}
\begin{itemize}
    \item Am Ende jedes Milestones soll ein Endprodukt entstehen,
    welches ausführbar und präsentierbar ist.
\end{itemize}
\end{frame}
\end{document}

